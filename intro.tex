\section{Introduction} \label{sec:intro}


%1. Interleaving - crash - explicit behaviour - 
Multi-threaded programs are hard to be correct. 
Concurrency bugs, including atomic violation, order violation, etc, are common in modern multi-threaded programs.
Extant works well explore mechanism of concurrency bug detectors, generation of interleavings that trigger concurrency bugs, and detecting bugs that explicitly cause execution order violation, wrong output and program crash
\cite{wu2015:collaborative,tsan,valgrind:pldi,lu:muvi:sosp,conseq:asplos11,conmem:asplos10}.
%and by employing memory sanitizers and thread sanitizers, severe concurrency bugs
%may be explicit(?..).

%2. Recent study - con attacks - structural , consequence, [in-explicit severer]
Recent studies\cite{acidrain:sigmod17,con:hotpar12} show rise of concerns about \emph{concurrency attacks}.
By triggering concurrency bugs, 
hackers may corrupt memory, and leverage the corrupted memory to conduct  
attacks including privilege escalations\cite{uselib-bug-12791,mysql-bug-14747}, hijacking code execution\cite{msiexploit}, bypassing security checks\cite{cve-2008-0034,cve-2010-0923,cve-2010-1754}, 
and breaking database integrity\cite{acidrain:sigmod17}.
These attack-causing bugs are often hidden in huge amount of concurrency bugs and cannot be easily distinguished. 
Concurrency attacks are cause by exploitable and harmful concurrency bugs, and will result in much more severe consequence than non-attack-causing bugs. 
A privilege escalation attack may run in background, drawing little attention of system administrator, 
but leave a permanent shell or backdoor hiding in the system, thus cause a long-term damage to security.
Also, beside knowing which subtle inputs can induce concurrency bugs, 
we also need to decide what  crafted inputs are needed for exploitation of concurrency bugs and launch of concurrency attacks. 



%3. Unfortunately, most extant work -- detect bug, corruption 
%   Our study - old bug - attack
%   However in danger
%   Sigmod 
Unfortunately, although great progress has been made, there does not exist a general model,
as well as practical and automatic tools for understanding and detecting concurrency attacks. 
Even some concurrency bugs has been detected and reported, 
professional may still miss the potential vulnerabilities caused by the bug. 
For instance, \emph{apache-25520}\cite{apache-bug-25520} has been 
reported over years and well studied by researchers\cite{lu:concurrency-bugs}.  
We are the first to exploit a new heap overflow attack leveraging on this bug and break the HTML integrity. 
This is dangerous because attacks may have been already exploited in wild. 
Hence, knowledge and detection of concurrency attacks is of crucial importance. 

%4. Two major challenges : 
%a. concurrent bug detectors (justify) 
%b. other input for attack
We studied \nattacks concurrency attacks and find two major challenges for detecting concurrency attacks. 
First, it's hard to distinguish the exact concurrency bug report which can conduct attack exploitation. 
Concurrency attacks are hidden in huge amount concurrency bug reports. 
For instance, a popular data race detector \tsan generates 1123 
race reports running \mysql's benchmark. However, 98\% reports are benign race 
and only 2 of rest data races are vulnerable and finally conduct 2 concurrency attack. 
Current concurrency bug detection tools are not designed to  
analyze whether a concurrency bug is vulnerable and exploitable. 
Hence developers need lots of efforts distinguishing and analyzing the reports given by concurrency bug detection tools. 
In practice, this is not compatible with today's software development. 
%ConMem\cite{conmem:asplos10} first propose to 
%consider concurrency bugs that may cause severe consequences (\eg program crash), 
%but our observation shows that explicit error (\eg program crash) is not necessary for concurrency 
%attacks.
%ConSeq\cite{conseq:asplos11} care about severe consequences and do intra-procedure analysis to help diagnose concurrency bugs. 
%However, propagation path ... .
%Here the propagation path?
%
%Worse still, crash bugs may even lead to more severe vulnerabilities. 
%In \emph{CVE-2017-7533} conducted by our team, although the data race primarily causes kernel crash, 
%we crafted the input and successfully conduct a privilege escalation attack without crashing the kernel.
%Current concurrency bug detecting tools are not designed to analyze this kind of latent vulnerabilities, 
%and hence may direct wrong level of warnings towards the bugs.  


Second, extant work ignores indicating extra inputs to conduct concurrency attacks.
A concurrency bug may become much more vulnerable when attackers 1.craft inputs to trigger the bug; 
2.employee other inputs running to construct their attacks. 
In CVE-2017-7533, we first leverage two crafted inputs running on two threads to trigger a data race and construct kernel heap overflow. 
We also require another inputs running on \emph{victim thread} to lay the target structure on the same heap. 
By corrupting the target structure, we finally achieve arbitrary code execution and get a root shell. 
Automatically indicating the inputs that construct concurrency attacks 
would be of vital helpful for developers to better understand the latent vulnerabilities.
%For example, in the Apache-25520 case, after knowing about information of victim thread, 
%we successfully increased the severity of the bug and conducted an integrity violation.

%5. New model for general attacks [] -- *vulnerability window 

To address the two challenges, we present a general model(\S\ref{sec:model}) 
for understanding concurrency attacks.
The model breaks down concurrency attacks into three stages: bug happening, 
bug-to-attack propagation, and attack happening. 
In this model, 
a concurrency bugs is triggered by bug-induced inputs. Then corrupted memory 
propagates through control flow and data flow of program execution. 
When corrupted memory propagates to vulnerable sites, a concurrency attack may be exploitable. 
This procedure addresses the first challenge. If we can provide hints for verified concurrency bug happening, 
and bug-to-attack propagation analysis, 
then developers can receive early warnings on concurrency attacks.

In special case, 
during data flow propagation, a buffer overflow may happen and additional threads' memory can be corrupted.  
In traditional sequential attacks, attackers can leverage buffer overflow to construct attacks like 
code hijacking, ... \cite{}. In concurrency situation, attackers can still conduct buffer overflow attack.  
However, indicating the other inputs is key for the second challenge.   
In sum, this model covers \nattacks concurrency attacks we studied. 


%6. Leveraging this model, framework - address two challenges.
Leveraging the model, we designed a practical, scalable and inter-procedural concurrency 
attack detection framework(\S\ref{sec:archi}), \xxx. 
The framework contain two phases.
%7. Two phase. 
The first phase is \emph{concurrency analyzer} to detect concurrency attacks, 
which augment current concurrency bug detectors.  
We provide two extra hints for explore concurrency attacks.
One hint is the benign schedules. The benign reports caused by adhoc
synchronizations have already implied how these synchronizations
act and how they work out schedules. Therefore, we
can use static analysis to extract these synchronizations from
the reports, automatically annotate these synchronizations in
a program, then we can greatly prune out these benign schedules
and their reports. The other hint is the bug-to-attack propagations, which
imply vulnerable inputs. Our study found that most vulnerable
races are already included in the race detectors’ reports
(\S\ref{label}), and concurrency attacks sites are often explicit in program
code (\S\ref{label}). Therefore, we can perform static analysis
on only the data and control flow propagations between the
bug reports and the potential attack sites, then we can collect
relevant call stacks and branch statements as the potentially
vulnerable input hints.
%We easily employ existing race detectors, and design a benign schedule reducer to reduce race reports(\S\ref{label}). 
%Then we use a dynamic verifier to verify whether a race can really happen. 
%After that, we do static analysis on the program 

[TODO]The second phase is \emph{concurrency fuzzer?} to provide another hint to 
indicate extra inputs for constructing a concurrency attack. 
The key goal of this phase is to find the input playing the victim role of 
concurrency attack. For instance, in CVE-2017-7533, we take a input that 
act as victim. And construct a kernel heap overflow to corrupt the victim structure's 
memory. By corrupting the memory, we successfully detected a arbitrary code 
execution vulnerability. Usually, the victim structure appearing on the heap
 is allocated with same memory allocation function. Hence we leverage this assumption, 
 and indicate the victim inputs. 
 


%9. Implementation / CVE . reduction
We implemented \xxx on Linux, supporting both user space and kernel space concurrency attack detection. 
\xxx adopts several race detectors including \tsan, \valgrind for user space and 
\ktsan, \ski for kernel space. We evaluated \xxx on 6 diverse, widely used programs, including \apache, \chrome, \libsafe, \linux, \mysql,
and \ssdb. \xxx’s benign schedule hints and runtime verifiers reduced 94.3\% of the race reports, 
and it did not miss the evaluated concurrency attacks. With the greatly reduced reports,
\xxx’s vulnerable input hints helped us identify subtle
vulnerable inputs, leading to the detection of 7 known concurrency
attacks as well as 4 previous unknown, severe ones
in \ssdb and \apache. The analysis performance of \xxx was
reasonable for in-house testing.


%10. Contribution
%     new model/ system / General -extension- .. new CVE..
This paper makes two major contributions:

\begin{tightenum}
\item \textbf{A general model for understanding concurrency attacks.} 
This model explains most concurrency attacks in wild and 
providing two major direction for detecting concurrency attacks. 
	
\item \textbf{A practical concurrency attack detection framework and its implementation, \xxx.} 
\xxx can easily employ existing concurrent bug detectors and vulnerability analyzer 
to improve the accuracy and ??? of detection.
	
\end{tightenum}

 

%11. The rest of section
The rest of this paper is structured as follows. 
\S\ref{sec:background} introduces the background of concurrency attacks.
\S\ref{sec:overview} gives an overview on the concurrency attack model and architecture of \xxx.
\S\ref{sec:framework} describes the design of \xxx. 
\S\ref{sec:implementation} states the implementation of \xxx.
In \S\ref{sec:discuss}, we discuss our limitations and future work. 
We evaluated \xxx and show results in \S\ref{sec:evaluation}.
We talk about related work in \S\ref{sec:related} and make a conclusion in \S\ref{sec:conclusion}.





