\it Just like bugs in single-threaded programs can lead to vulnerabilities, 
bugs in multithreaded programs can also lead to concurrency attacks. 
Unfortunately, there is little quantitative data on how well existing tools 
can detect these attacks. This paper presents the first quantitative study 
on concurrency attacks and their implications on tools. Our study on 
\nprog widely used programs reveals \nattacks concurrency attacks with 
broad threats (\eg, OS privilege escalation), and we built scripts to 
successfully exploit \nreproduced attacks. Our study further reveals that, only 
extremely small portions of inputs and thread interleavings (or 
schedules) can trigger these attacks, and existing concurrency bug detectors 
work poorly because they lack help to identify the vulnerable inputs and 
schedules.

% Although existing 
% concurrency bug detection tools can detect the vulnerable concurrency bugs, the 
% excessive reports of these tools bury the bugs and their attacks.

% Even if people hit the vulnerable reports by 
% luck, these reports only have the bugs' structual patterns (\eg, thread 
% interleavings), but have no idication on whether these bugs may lead to 
% concurrency attacks.

Our key insight is that the reports in existing detectors have implied 
moderate hints on what inputs and schedules will likely lead to attacks and 
what will not (\eg, benign bug reports). With this insight, this paper presents 
a new directed concurrency attack detection approach and its 
implementation, \xxx. It extracts hints from the reports with static 
analysis, augments existing detectors by pruning out the benign inputs and 
schedules, and then directs detectors and its own runtime vulnerability 
verifiers to work on the remaining, likely vulnerable inputs and schedules.

Evaluation shows that \xxx reduced \reducerate reports caused by benign inputs 
or schedules and detected \nknownVul known concurrency attacks. \xxx also 
detected \nunknownVul previously unknown concurrency attacks, including a 
use-after-free attack in \ssdb, and an integer overflow and HTML integrity 
violation in \apache. All \xxx source code, exploit scripts, and results 
are available at \github.

% This paper presents \xxx, an analysis framework to detect concurrency 
% attacks. \xxx first runs concurrency bug detection tools to generate bug 
% reports, it then provides four types of general filters to prune the benign 
% reports, and it finally runs our analysis tool to indicate whether the remaining 
% reports may lead to concurrency attacks. \xxx's analysis tool achieves 
% reasonable scalability and accuracy, two difficult program analysis 
% challenges, by leveraging bugs' dynamic information (\eg, call stacks) to 
% direct static analysis toward vulnerable program paths.
