\section{Related Work} \label{sec:related}

% Initially from Work by Ruigu

% TOCTTOU attacks
\textbf{TOCTTOU attacks.} Time-Of-Check-to-Time-Of-Use attacks \cite{bishop1996checking,tsyrklevich2003dynamic, tsafrir2008portably,wei2005tocttou}
target mainly the file interface, and
leverage atomicity violation on time-of-check (access())
and time-of-use (open()) of a file to gain illegal file access. 

A prior concurrency attack study \cite{yang2016concurrency} elaborates that concurrency
attacks are much broader and more difficult to track
than TOCTTOU attacks for two main reasons. First, TOCTTOU
mainly causes illegal file access, while concurrency attacks
can cause a much broader range of security vulnerabilities,
ranging from gaining root privileges \cite{uselib-bug-12791} , injecting malicious
code \cite{freebsdcve}, to corrupting critical memory \cite{apache-bug-25520}. Second, concurrency
attacks stem from miscellaneous kinds of memory access, and
TOCTTOU stem from file accesses, thus handling concurrency
attacks is much more difficult than TOCTTOU.

Another prior study \cite{concurrency:woot07} further
defined two more kinds of vulnerabilities: One is \textbf{TOATTOU}(Time-Of-Check-to-Time-Of-Use),
in which the audit log diverges due to non-atomicity so that attackers could mask activities to avoid IDS triggering; 
The other is \textbf{TORTTOU}(Time-Of-Check-to-Time-Of-Use), unique to system call wrappers, in which attackers could 
modify system call arguments after the wrapper has replaced the arguments but before the kernel accesses them.

% Sequential Security Techniques
\noindent
\textbf{Sequential security techniques.} Defense techniques for sequential
programs are well studied, including taint tracking \cite{taintdroid:osdi10,lift:micro06,sospinfomationflowcontrol,valgrind:pldi}, 
anomaly detection \cite{taskrecycling:ppopp90,schonberg:pldi89}, address space
randomization \cite{aslr-fail:ccs04}, and static analysis \cite{seth:pldi,engler:vmcai,wagner:intrusion,coverity:cacm,klee:osdi08}.

However, with the presence of multithreading, most existing
sequential defense tools can be largely weakened or even
completely bypassed \cite{con-tr}. For instance, concurrency bugs
in global memory may corrupt metadata tags in metadata
tracking techniques. Anomaly detection is lack of a concurrency
model to reason about concurrency bugs and attacks.

% Concurrency Reliability Tools
\noindent
\textbf{Concurrency reliability tools.} Various prior systems work
on concurrency bug detection \cite{yu:racetrack:sosp,savage:eraser,racerx:sosp03,lu:muvi:sosp,avio:asplos06,conmem:asplos10,conseq:asplos11,kasikci2013racemob,wester:parallelizing:asplos13}, diagnosis \cite{racefuzzer:pldi08,ctrigger:asplos09,atomfuzzer:fse08,attariyan2012x,failure:sketch:sosp15}, and correction \cite{deadlock-immunity:osdi08,gadara:osdi08,wu:loom:osdi10,automated:osdi}. They focus on concurrency bugs themselves, while
OWL focuses on security related consequences of concurrency
bugs. Therefore, these systems are complementary to OWL.

Conseq \cite{conseq:asplos11} detects harmful concurrency bugs by analyzing
their failure consequence. Its key observation is that concurrency
bugs and those bugs' failure sites are usually within a
short control and data flow propagation distance (e.g., within
the same function). Concurrency attacks(targets of OWL)
usually exploit corrupted memory that resides in different
functions, thus Conseq is inadequate for concurrency attacks.
Though, Conseq’s proactive harmful schedule exploration technique
will be useful for OWL to trigger more vulnerable schedules.

% Static Vulnerability Detection Tools
\noindent
\textbf{Static vulnerability detection tools.} There are already a variety
of static vulnerability detection approaches \cite{livshits05finding,yamaguchi2014modeling,felmetsger2010toward,felmetsger2010toward,srivastava2011security,zhang:usenix:sec02}, which fall into two categories based
on whether they target general or specific programs.

The first category \cite{livshits05finding,yamaguchi2014modeling} targets general programs and
these approaches have been shown to find severe vulnerabilities
in large code. However, these pure static analyses may
not be adequate to cope with concurrency attacks. Benjamin
et al. \cite{livshits05finding} leverage pointer analysis to detect data flows from
unchecked inputs to sensitive sites. This approach ignores
control flow and thus it is not suitable to track concurrency
attacks like the Libsafe one in §4.3\textbf{[TOCHECK]}. Yamaguchi et al. \cite{yamaguchi2014modeling}
did not incorporate inter-procedural analysis and thus is not
suitable to track concurrency attacks either. Moreover, these
general approaches are not designed to reason about concurrent
behaviors (e.g., \cite{yamaguchi2014modeling} can not detect data races).

OWL belongs to the first category because it targets general
programs. Unlike the prior approaches in this category,
OWL incorporates concurrency bug detectors to reason about
concurrent behaviors, and OWL’s consequence analyzer integrates
critical dynamic information (i.e., call stacks) into
static analysis to enable comprehensive data-flow, control-
flow, and inter-procedural analysis features.

The second category \cite{felmetsger2010toward,arzt2014flowdroid,srivastava2011security,zhang:usenix:sec02} makes static analysis
focused on specific behaviors (e.g., APIs) in specific programs
to achieve scalability and accuracy. These approaches check
web application logic \cite{felmetsger2010toward}, Android applications \cite{arzt2014flowdroid}, cross
checking security APIs \cite{srivastava2011security}, and Linux Security
Module \cite{zhang:usenix:sec02}. OWL’s analysis is complementary to these
approaches; OWL can be further integrated with these approaches
to track concurrency attacks.

% Symbolic Execution
\noindent
\textbf{Scheduler Control and Symbolic execution.} Scheduler control 
is a way to exploiting synchronization bugs by using interrupting and scheduling threads.
AsyncShock \cite{weichbrodt2016asyncshock} is such a tool for thread manipulation. Scheduler control can be utilized by OWL to 
find thread execution order of triggering concurrency attacks.
Symbolic execution is an advanced program
analysis technique that can systematically explore a program’s
execution paths to find bugs. Researchers have built
scalable and effective symbolic execution systems to detect
software bugs \cite{dart:pldi,cute:fse,godefroid:grammar-fuzzing,godefroid:whitebox-fuzzing,klee:osdi08,yang:malicious-disk:oakland06,cadar:exe:ccs06,s2e:hotdep09,taas:socc10,ramos2015under}, block
malicious inputs \cite{costa:openbsd}, preserve privacy in error reports \cite{castro:bug-report-privacy},
and detect programming rule violations \cite{woodpecker:asplos13}. Specifically,
UCKLEE \cite{ramos2015under} has been shown to effectively detect hundreds
of security vulnerabilities in widely used programs. Symbolic
execution is orthogonal to OWL; it can augment OWL’s input
hints by automatically generating concrete vulnerable inputs.